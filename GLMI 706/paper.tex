\documentclass[12pt,a4paper]{article}

% Basic package imports
\usepackage[utf8]{inputenc}
\usepackage[margin=1in]{geometry}
\usepackage{setspace}
\usepackage{natbib}
\usepackage{hyperref}

% Page settings
\doublespacing

% Hyperlink settings
\hypersetup{
    colorlinks=true,
    citecolor=black,
    urlcolor=blue
}

\begin{document}

\begin{center}
\textbf{\LARGE Learning Reflection}
\end{center}

\vspace{0.5cm}

Last month's mock interview threw me into an unprecedented state of confusion. The interviewer asked a question I had already answered countless times: ``Why did you switch from an insurance major to IT?'' I began speaking almost automatically, fluently telling a story about how I ``discovered during an internship at an insurance company that I was more passionate about data analysis than about sales.'' After the interview, a classmate praised me for how ``well-crafted'' my story was, but I suddenly froze---just the day before, at an alumni networking event, I had told a senior a different version: ``I have always been interested in technology; I just did not understand myself well when I chose my major.'' And on the phone with my parents, my explanation had become, ``The IT industry has better prospects and makes immigration easier.'' Three different stories, three different ``selves.'' In that moment, I asked myself in earnest for the first time: Which of these is actually the real me? Or have I already lost myself in these carefully woven narratives?

I quickly passed a simple moral judgment on myself: I was not ``sincere'' enough. In my previous imagination, a ``mature professional'' should have a stable, almost recitable, and logically consistent answer to this question, able to withstand probing in any situation. Yet the current me feels more like an actor constantly changing makeup on different stages.

However, this semester's readings offered a new perspective: perhaps I am not being ``inauthentic'' but am instead undergoing a typical ``macro work role transition.'' As \citet{Ibarra2010Identity} describe, this transition requires weaving stories that connect past experiences, present decisions, and future goals, so that both they and others can believe that ``this transition makes sense.'' From this perspective, those three seemingly contradictory accounts may not be ``lies,'' but rather three experimental narratives of the same career transition. In each case, I was trying to persuade not only my listeners but also myself.

Looking more closely, I was instinctively adjusting my narrative for different ``audiences'': for the interviewer, ``transferable skills''; for the alumnus, an ``emotional connection'' to my previous industry; and for my parents, ``future prospects'' to reduce their anxiety. Each telling was in pursuit of a kind of situated ``reasonableness.'' The problem was that, even as I adapted my stories outwardly, I remained inwardly attached to the assumption that there must be one single true reason---as if admitting that my motives were mixed and even contradictory would render me, as a person, untrustworthy. This was precisely the source of the intense discomfort I felt after the mock interview.

A study on ``provisional selves'' by \citet{Ibarra1999Provisional} has led me to understand this discomfort in a completely different way. Their research suggests newcomers experiment with ``provisional selves''—try-on versions of possible future identities—rather than possessing a stable identity from the start. From this perspective, my three versions are not a betrayal of some ``real'' self, but rather cautious probes into three different possible future IT selves: Which version flows more smoothly? Which one elicits more recognition? Which one makes me feel more empowered, rather than guilty, when I say it aloud?

Looking back over my recent experiences, I have noticed that I have already entered an ``identity learning cycle.'' According to \citet{Pratt2006Constructing}, individuals customize their professional identities through repeated loops of ``concrete work experience--felt sense of identity--interpretation of meaning--subsequent action.'' For me at this stage, the ``work setting'' has temporarily become CV revisions, mock interviews, networking events, and conversations with my supervisor. Each time I tell the story of my career transition, it functions as a ``work episode''; each wave of awkwardness, relief, or guilt that follows is emotional feedback about what kind of professional I am becoming. And the process of writing this reflection is, in itself, an act of assigning new meaning to these experiences.

The real turning point came when I realized that I neither needed nor could possibly find a single, uniquely ``true'' causal chain to explain my decision to move from insurance to IT. As \citet{Savickas2005CareerConstruction} argues, in a highly uncertain labor market, a career is less like a linear ladder and more like a story that is continually being rewritten and reinterpreted. In this story, each segment of the past is not a static fact but material that can be rearranged and reconnected. I used to treat my background in insurance as having taken a long detour, and I even subconsciously downplayed it in interviews. Yet if I allow myself to re-author the story, that experience can be rewritten as: how I learned, in a highly regulated and risk-sensitive industry, to understand complex products and uncertainty, and how I began to think about using technology to improve these processes---this is not fabrication, but simply another way of organizing the facts.

After realizing this, I began to treat my recent behaviors as a series of designable and observable ``identity experiments,'' rather than ``self-presentation events'' to be judged as true or false. I plan to deliberately document the different versions of my answer to the ``Why IT?'' question, together with the audience, setting, and my felt experience. Every few weeks, I will review these records and ask myself: Which narrative best reflects the values I genuinely care about? Which one sounds most natural in my own voice, rather than like reciting a script?

At the same time, I hope that, in my upcoming internship applications and project choices, I can more deliberately turn these ``possible selves'' into ``testable hypotheses.'' For example, if one ideal self is ``an IT professional who serves as a bridge between business and technology,'' I will prioritize roles that require both, in order to test whether I truly enjoy such a role. If another ideal self is ``a more research-oriented IT practitioner,'' I will take on small projects that involve reading papers and designing experiments, and observe my energy level in this mode of work~\citep{Markus1986Possible,Pratt2006Constructing}. In this way, I am not merely fantasizing about ``what kind of person I want to be in the future,'' but using concrete choices and actions to move, step by step, closer to the \emph{possible selves} I am genuinely willing to become.

Returning to that mock interview, if I were asked again now, ``Why did you move from insurance to IT?'', my answer would be noticeably different. I would no longer try to let a single ``real reason'' shoulder all the complexity; instead, I would more candidly acknowledge that this choice arose from my interest in technology, my disappointment with existing work patterns in insurance, and pragmatic considerations about future career prospects. Rather than trying to prove that I have been ``single-minded'' all along, I would seek to show---both to the interviewer and to myself---that I am taking this macro role transition seriously and am carefully crafting a career story that can both connect my past and make room for my future.

% References
\bibliographystyle{apalike}
\bibliography{references}

\end{document}