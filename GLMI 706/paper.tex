\documentclass[12pt,a4paper]{article}

% Basic package imports
\usepackage[utf8]{inputenc}
\usepackage[margin=1in]{geometry}
\usepackage{setspace}
\usepackage{natbib}
\usepackage{hyperref}

% Page settings
\doublespacing

% Hyperlink settings
\hypersetup{
    colorlinks=true,
    citecolor=black,
    urlcolor=blue
}

\begin{document}

\begin{center}
\textbf{\LARGE Learning Reflection}
\end{center}

\vspace{0.5cm}

% Write your reflection content here
As a master's student in IT, I once took deep pride in my programming abilities. Over the past few years, I had devoted nearly all my spare time to the world of code---from basic syntax to various libraries, from late-night bugs to programs that finally ran at dawn. These skills were not only my professional foundation but also formed the core of my professional identity. However, a series of experiences in recent months has completely transformed my understanding of the concept of ``skills.'' 

It all began with a seemingly ordinary project. I needed to integrate \texttt{Unicorn.js}---a CPU emulator---into the frontend. Considering the project's complexity and time pressure, I decided to try a tool recommended by a classmate: Claude Code. Initially, I approached it with just a ``let's-see'' attitude, since as a programmer who considered myself experienced, I was fully confident in my ability to write code by hand.

The experience of using Claude Code exceeded my expectations. What had originally been estimated to take several days of integration work achieved basic functionality in just a few hours with AI assistance. It not only understood complex technical requirements but also predicted potential issues and provided optimization suggestions. Watching the code rapidly generate on screen, I experienced a complex emotion---both excitement about the efficiency gains and an indescribable unease.

This sense of unease gradually deepened. Whenever I saw AI effortlessly complete code that I once needed to think through repeatedly to write, a voice echoed in my mind: ``If AI can so easily do my work, what's the point of all my years of effort?'' This thought hung over me like a shadow, making me feel deeply threatened even while using this powerful tool. I even began deliberately avoiding AI, trying to prove my value by writing all code by hand.

The turning point came during the fourth week of coursework. When I read ``The Rise of Technology and the Impact on Skills'' by Ra et al., the discussion of \emph{learnability} deeply resonated with me. They emphasize that in a rapidly changing technological environment, what truly matters is not mastering specific skills, but having the willingness and ability to learn, unlearn, and relearn \citep{ra2019technology}. This concept forced me to reflect: Where exactly did my unease come from?

On the surface, I was worried that AI would replace me. But upon deeper reflection, I found the source of this unease was more complex. I realized that my learning experience over the past few years had created an illusion---that once I mastered programming skills, I could rely on them once and for all. I had spent enormous time and energy learning various programming languages, frameworks, and algorithms, subconsciously expecting these investments to bring lasting returns. When AI appeared and easily completed these tasks, what I felt was not just the threat of skill devaluation, but also reluctance about my past investments being ``sunk.''

This made me see another side of myself---resistance to learning new things. Although I had always prided myself on ``loving to learn,'' in reality, I only loved learning things that could consolidate my existing advantages. Faced with an AI collaboration model that required repositioning myself and changing my work methods, my first reaction was not to embrace it, but to resist. This resistance wore the cloak of ``professional spirit''---I told myself that writing code by hand was real technology, and relying on AI was laziness.

This realization shocked me. It turned out that my attachment to old skills was essentially a form of learning inertia. I hoped to gain permanent career security through a one-time learning investment. How naive this thinking appears in today’s rapidly iterating technology landscape. As Ra et al. point out, the ``shelf life'' of professional skills is getting shorter and shorter. Trying to gain career security by mastering a specific skill is like trying to respond to a constantly changing world with a static posture \citep{ra2019technology}.

After recognizing this, I began to re-examine my interaction with Claude Code. I found that when I let go of my attachment to ``pure, handcrafted coding,'' I actually opened up an entirely new learning domain. Collaborating with AI required me to learn new skills: how to accurately describe requirements so that AI could understand them, how to verify and optimize AI-generated code, and how to handle more complex system problems with AI’s help. These were all areas I had never touched before.

The shift in testing strategy particularly made me appreciate the true meaning of \emph{relearn}. Previously, I firmly believed that line-by-line code review was the best way to ensure quality---this had almost become my professional creed. But after using AI to generate large amounts of code, I experimented with a new approach, shifting toward \emph{test-driven development}---verifying system behavior through comprehensive acceptance tests and end-to-end tests. Initially, abandoning line-by-line review made me uneasy, but gradually I found this method not only more efficient but also freeing: it pulled me out of the morass of implementation details and let me think about system design and business value from a higher level.

In the later stages of the project, my relationship with Claude Code underwent a fundamental transformation. It was no longer a threat, but a partner. We formed an efficient collaboration model: I was responsible for understanding requirements, designing architecture, and formulating strategies; AI handled rapid implementation and initial optimization; then I performed verification, adjustment, and integration. This collaboration allowed me to handle more complex problems than before, think about higher-level design, and deliver more valuable outcomes.

Looking ahead, I plan to completely change my learning strategy. First, I will regularly review my skill portfolio, identifying which skills are depreciating and which are worth deepening. Second, I will maintain curiosity about new things, actively exploring rather than instinctively rejecting. Most importantly, I will cultivate what \citet{ra2019technology} call the ``meta-ability of learning''---not just learning specific technologies, but learning how to quickly master new domains, how to judge learning directions, and how to maintain balance amid change.

This is not a story about ``human vs.\ machine,'' but a choice between ``growth vs.\ stagnation.'' I choose growth, choose to dance with AI, choose to find certain value in uncertainty. Because I finally understand that in this rapidly changing era, the only certainty is change itself, and the best way to cope with change is to become part of it.

% References
\bibliographystyle{apalike}
\bibliography{references}

\end{document}
