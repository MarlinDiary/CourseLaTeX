\documentclass[12pt,a4paper]{article}

% Basic package imports
\usepackage[utf8]{inputenc}
\usepackage[margin=1in]{geometry}
\usepackage{setspace}
\usepackage{natbib}
\usepackage{hyperref}

% Page settings
\doublespacing

% Hyperlink settings
\hypersetup{
    colorlinks=true,
    citecolor=black,
    urlcolor=blue
}

\begin{document}

\begin{center}
\textbf{\LARGE Learning Reflection}
\end{center}

\vspace{0.5cm}

I fixed a stubborn concurrency bug and designed what I believed was an elegant thread communication scheme. I let out a long breath, lifted my head from the screen with the quiet satisfaction of success, and looked around---and the sense of achievement vanished almost instantly. My supervisor was busy, and my classmates were all struggling with their own projects. The hard problem I had solved and the elegant architecture I had built were known to no one, and cared about by no one.

In that moment, I fell into a profound crisis of meaning: if technical success could not give me a sense of meaning, then where does meaning actually come from?

For a long time, like many of my classmates in computer science, I equated the meaning of work with solving one coding challenge after another. The independent research project I took on this semester---developing an assembly language teaching tool for an undergraduate computer systems course---seemed perfectly tailored to that belief. It offered endless technical challenges, and it gave me a high level of autonomy to take full responsibility for everything from architectural design to implementation. By all accounts, this should have been a perfect and meaningful piece of work.

In the first few weeks of the project, I was completely immersed in pure technical problem-solving. But as the project went on, the feeling of being unseen grew stronger. The harder I worked and the more technical progress I made, the more I felt a deep sense of frustration and isolation. My work had turned into a one-person battle. This sense of defeat reached its peak in the second half of the semester, when I realized I was experiencing the classic symptoms of job burnout described by \citet{Maslach2001Burnout}, especially a diminished sense of personal accomplishment. Even though the project was technically successful, it felt meaningless to me, and for the first time I began to seriously question myself: ``Am I actually suited to become the kind of researcher I once imagined? If this is what research really looks like---solving problems alone that no one cares about---why should I keep going?''

Looking back now, I can see that I had fallen into a classic contradiction: I had a very high level of autonomy in my work and I was making clear technical progress, yet I still felt that none of it was meaningful. The framework by \citet{Laaser2022MeaningfulWork}, which I read this semester, offered a precise diagnosis. They argue that the experience of meaningful work is built on ``autonomy, dignity, and recognition,'' and that each of these dimensions has both an objective and a subjective aspect. My problem was this: I had extremely high objective autonomy, but I was almost completely lacking in recognition.

At the moment when I was beginning to seriously doubt my academic ambitions, the paper by \citet{Bailey2017TimeReclaimed} on temporality and the experience of meaningful work illuminated my situation. Their research shows that meaningful work is not a continuous, stable state; instead, it tends to appear in specific moments of temporal transcendence. This made me realize that my problem was that I had mistakenly assumed that meaning must be obtained in the present. I had trapped myself in what \citet{Bailey2017TimeReclaimed} describe as a ``lost present''---a present cut off from both past and future, made up only of technical details and lines of code. All of my frustration came from expecting that my effort in the present would immediately produce recognition in the present.

\citet{Bailey2017TimeReclaimed} also offered me a crucial way out. Their insights about academic workers were especially striking: how do scholars find meaning in research that no one understands in the present? They do so because they believe that this work may have impact in the future. This described my situation exactly. I suddenly understood: the meaning of my work was never meant to exist in the present. All of its meaning is anchored in the future. The value of building this teaching tool is not in how elegantly I solve technical problems right now, but in how it may one day help students---beneficiaries I have not even met yet---understand assembly language more clearly. The entire value of my work is condensed into that future moment of contribution.

But why did this idea of future students bring such strong comfort to me in the present? The work of \citet{Grant2007RelationalJobDesign,Grant2008TaskSignificance} gave me the answer: the sense of meaningfulness in work does not primarily come from the task itself, but from perceiving that our work has a positive impact on other people. One of the strongest motivational forces is being able to feel connected to the beneficiaries of our work. At that time, however, my problem was that I was separated in time and space from my beneficiaries---the students. Because of that distance, my own sense of task significance was extremely low.

In the later stages of this research project, I deliberately engaged in a form of cognitive reframing for myself \citep{Wrzesniewski2001JobCrafting}. I no longer defined my work as a lonely technical development task, but instead reframed it as a valuable prosocial contribution. I was no longer just a programmer wrestling with a compiler; I became a builder of an educational tool. My identity shifted from an isolated technical specialist to someone connected to future students as a contributor. This aligns with \citet{Rosso2010MeaningOfWork}, who argue that one of the most powerful sources of meaningfulness in work is self-transcendence---serving a purpose larger than oneself.

This deep reflection did not make me abandon the goal of becoming a researcher. On the contrary, it gave me an unprecedented level of clarity and commitment to pursue it. I have come to understand that the crisis of meaning I experienced is a necessary passage for all knowledge workers. My core lesson is this: even when the meaning of work cannot be granted to us objectively, we still have the agency to construct it subjectively, especially through two dimensions---time and other people.

Based on this, I have developed the following plan of action for my future life as a researcher.

I must actively construct temporal transcendence. In any future long-term and solitary research project, I will make it explicit at the very beginning how the tasks I am doing right now connect, in writing, to the larger, future-oriented purpose of the work. This ``map of meaning'' will serve as a constant reminder of why I started.

I also need to build a feedback loop of recognition. I will actively create opportunities for work that would otherwise remain invisible to be seen and discussed. I will force myself to share my progress with my supervisor, my peers, and even non-technical friends. In addition, I will try to offer my tool to other courses in the department, or release it openly to a broader community, in order to deliberately seek recognition and, in doing so, inject meaning into my work on a wider stage.

% References
\bibliographystyle{apalike}
\bibliography{references}

\end{document}
